% Title: Ajahn Chah's Teachings on Nature
% Author: Ajahn Pasanno

Tomorrow I am invited to teach a day-long retreat at Spirit Rock
Meditation Center on the theme of ``Ajahn Chah's Teachings on Nature.''
For the past few days I've been preparing and have steeped myself in
Ajahn Chah's teachings -- swimming in the soup of his biography as well
as reading and listening to some of his talks. I've enjoyed it
immensely. I have no idea what will come out in this evening's talk, but
I think it will be influenced by the things that I've reviewed. 

I have many recollections of Ajahn Chah. I'm completely biased. I was an
early student of his, and I am a monk because of the inspiration Ajahn
Chah gave me. For everything I say, everything I've learned and the
practice I've done, I owe a great debt of gratitude to Ajahn Chah. His
teaching and his presence still affect me. 

One of Ajahn Chah's unique qualities as a teacher was his ability to
explain and encourage people in ways that made the practice very
tangible. Some of this was his ability to use imagery and similes. One
of the images that he gave of the practice was of a coconut tree. A
coconut tree draws nutriments from the planet; it draws elements good
and bad, clean and dirty, up through the roots and into the top of the
tree and then produces fruit that gives both sweet water and delicious
meat. 

In the same way, as practitioners, we take all the different experiences
that we have, all the different contacts with the world that we have, 
and we draw them up through our practice of \emph{Virtue}, of \emph{Concentration}, of
\emph{Wisdom}. They can be all transformed into something that is very
peaceful, that bears great fruit in terms of insight, understanding, and
a tremendous balance and sense of peace. We don't need to be shy or
worried or concerned about the different experiences that we have --
whether we're successful or not in our meditation, or whether we
experience \emph{praise} or \emph{blame}, \emph{gain} or \emph{loss}. 

All of those experiences can be drawn up, through our practice, through
our training. They can all be transformed. I think that's a wonderfully
encouraging image. 

Another image Ajahn Chah used for practicing meditation is the leaves in
the trees and the forest. Quite naturally, the leaves in the forest are
quite still. Only when the wind blows will the leaves vibrate or shake, 
be blown back and forth. In the same way, our mind, our actual mind, our
real mind, is always still and steady. It's the moods of the mind that
shake it. 

When the winds of our moods, impressions, thoughts and feelings come up, 
we take the mind to be the various moods and impressions, rather than
recognizing that it's just the winds of mood, of thought and feeling, of
perception. The underlying mind is the quality of knowing. The
underlying mind is the quality of being present. With that quality, we
are able to distinguish between the wind of mood and the quality of
knowing and able to be attentive, and recognize that both those things
are happening. The moods of the mind -- the impressions, the reactions, 
the additions that we make and the proliferations that we add -- affect
what we consider to be the mind. In fact, we misperceive experience or
don't recognize the distinction between the two. 

One doesn't stand outside and force the wind not to blow or get upset
because the wind does blow. It's just a natural phenomenon. In the same
way, we can allow the mind to become steady, to become peaceful, to
attend in ways that don't get caught up in the activity of the mind. Or, 
we can be swept up by the winds of change that blow through the mind, 
but see that as a natural phenomenon. Ajahn Chah was skillful at getting
us to really pay attention to the nature and naturalness of the practice
- that very natural reality we easily miss. 

So often we tend to believe that things should be special in some way, 
they should conform to some ideal or doctrinal position. But Ajahn Chah
was able to see through that habit, that human tendency. The Noble
Truths that the Buddha taught were about Nature. All of our experience
is something that's in Nature, it's something natural. But that truth is
something we overlook. Instead, we create all sorts of suffering and
confusion around it. 

One time when I was sitting with Ajahn Chah, I was asked to be a
translator for a visitor -- a journalist from Sweden. He was
interviewing various spiritual teachers and asking the same questions, 
and, of course, getting a huge range of answers. His questions included: 
``Why do you practice? How do you practice?  And what results do you get
from the practice?'' My participation as the translator complicated the
situation and created a big obstacle. I felt a particular irritation
towards the monk from Bangkok who brought the journalist to the
monastery. There were also my views and opinions about what I thought
were idiotic questions asked by the journalist. This made the situation
really interesting because \emph{nothing} slipped by Ajahn Chah. 

We sat down and the whole farcical scene started to play itself out. 
The journalist asked questions, then I translated them for Ajahn Chah. 
Ajahn Chah started talking about something else -- asking his own
questions and talking about this and that.  After some time, he turned
to me and asked, ``What were those questions again?'' I had to
re-translate them and then Ajahn Chah went off on another tangent. After
a while he said, ``Did the journalist ask some questions? Oh, what were
those questions?'' And then I had to translate the questions yet again
and, of course, Ajahn Chah went off again, and then asked, ``Has anybody
got a pencil and paper? Can somebody write those questions down for
me?'' So we went to find the pencil and paper. Ajahn Chah then asked, 
``So what was that first question?'' I had to translate the question
slowly enough so Ajahn Chah could write it.  ``Okay, \emph{why} do we
practice?'' Ajahn Chah wrote it down. ``What was that second question
again?'' ``\emph{How} do we practice?'' ``Oh, okay,'' and he wrote it down. 

 ``What was that third question?'' He wrote it down. Then he looked at
 the journalist and asked really sharply, ``\emph{Why do you eat?}''

That question took the journalist aback, and he responded, ``Uhh\ldots{}
I'm not quite sure.''

``No, why do you eat?'' Ajahn Chah said, ``I want an answer to the
question, why do you eat?''

The journalist responded, ``I eat because I'm hungry.'' And Ajahn Chah
said, ``Exactly -- that's why we practice. We're hungry -- we're hungry
for truth, we're hungry for peace, we're hungry for reality. We are
suffering, and we're hungry to be able to free ourselves from
suffering.'' And then he talked on that theme, explaining that when you
really realize you're hungry, you look around and try to find ways of
practice that make sense to you. And the result is that if you are
hungry and you find something to eat, and you find out how to make that
food and nourish yourself, you will be full; you will be replete; you
will be at ease.  And that is the whole purpose of practice. 

He put it into something very immediate, natural, and practical, rather
than a theoretical and doctrinal position. And he gave me a bad time
while he did it. It was very masterful, and he was so skillful at doing
that, at picking up on things. He used many different ways of
encouraging people in practice.  He wasn't fixed in his techniques or
his methodologies. He encouraged people to practice and to reap the
fruits of the practice. When he talked about meditation, meditation
techniques and tools, he would be very open. 

When Ajahn Sumedho first went to Wat Nong Pa Pong, he was the first
Westerner Ajahn Chah had ever seen; he had not really taught anybody
other than local villagers. Ajahn Sumedho had been practicing at a
meditation center, where he was a novice for a year; he had recently
ordained as a monk. The center was focused on the Mahasi Sayadaw --
technique -- a Burmese method of ``walking, sitting, walking, sitting.''
Ajahn Sumedho found himself getting quite dry. Then he experimented with
a technique from the Chan tradition, with translations of a Chan
meditation retreat that Master Hsu Yun had given in Hong Kong.
When that retreat was given, Master Hsu Yun was about 115 years old and still leading
meditation retreats. Master Hsu Yun used a completely different
methodology than Ajahn Chah was used to, the \emph{Hua-tou} method of
questioning -- posing the question, ``Who am I?'' or something like it
-- trying to come back to the source of the \emph{knowing}. 

Ajahn Sumedho asked Ajahn Chah if he could use this technique, and if he
had to follow a particular method. Ajahn Chah asked him what he was
doing, what results he was getting, and how he applied it, and said, 
``Yeah, if it's working, fine.'' He had that kind of openness to
different ways of practice and encouraged people to experiment. He
compared practice with paying attention to the food you eat. Some foods
will upset your stomach; some foods will give you energy, while some
foods will make you sluggish. Some food might taste good but may not be
good for you, or might not taste good but be nourishing for you.  In the
same way that you have to pay attention to the result of the food that
you eat, you have to pay attention to different methods, techniques and
ways of practice. You have to see what the flavor is, what the results
are, what the benefits and drawbacks are. 

Ajahn Chah had the sense that there is a pool of options and
opportunities that we have to learn how to apply skillfully.  Practice
is not ``by the book, this one method is going to work for everybody.''
Ajahn Chah's approach to practice was not to just ``dig in'' to a
technique, meditation or training and ``put your foot to the pedal'' and
go to the end of it. It's not a sprint; it's more like a marathon. You
have to be able to pace yourself and be in it for the long haul. You
have to be ready to gauge how to sustain practice, how to have
continuity of practice, how to make the continuous effort in practice. 
Continuous effort is not a ``striving and pushing'' effort. It's a
\emph{sustaining} effort -- a continuity of attention, reflection, application
-- because that's what really undermines the habitual tendencies, the
defilements, \emph{ignorance} and \emph{delusion}. Steadiness and continuity allow the
practice to unfold and to reveal what we need to let go of; what we need
to develop. That's the essence of the practice. 

This sense of the naturalness of practice is why Ajahn Chah put a lot of
emphasis on \emph{Virtue}. The things that he would emphasize most were
Virtue and Right Understanding, \emph{Sīla} and \emph{Sammā-diṭṭhi} -- meditation
grows from that foundation. Wisdom and Penetration rise up out of the
foundation of Virtue and Right Understanding, from being attentive to
those qualities.  There's a naturalness that's not about technique or
about heroic effort for a short period of time, but really knowing how
to be rounded and grounded in the practice. He exemplified that in his
own being, his own commitment to virtue and integrity. He was
impeccable, but he was never forced. You never got the sense of him
worrying about his precepts or about his conduct. Everything was steeped
within him, and the expression of his life and his being was as a person
who had tremendous integrity and virtue. In the same way, his wisdom -
his discernment -- weren't thought out or planned, and he didn't just
recite it back to you. It arose out of right view and right
understanding. He had contemplated, cultivated, and investigated, 
questioned and developed it over the time of his training. That's what
he encouraged in all of us, to be willing to put in the time and the
effort, to be consistent in the practice, to have that continuity of
training. 

These are some of the reasons why he wouldn't ordain people quickly. It
was quite rare in his time to go against the custom of temporary
ordination or receiving ordination quickly.  With Ajahn Chah, you'd have
to be an \emph{Anāgārika} for a year, then a novice for a year, and you would
have to stay with him as a monk for five years. It was quite rare that a
monk and a teacher emphasized that level of commitment. People would
complain, ``Why does it have to be so difficult? Why can't a person take
ordination more quickly?'' Ajahn Chah responded, ``People take
ordination quickly, and then they disrobe quickly.'' If it's too easy to
ordain, then it's easy to leave and go off somewhere else. If there's
too much wandering around and doing too many other things, you don't get
it; you don't reap the fruits of the practice. 

You have to be willing to make a commitment to the training, to develop
that continuity and consistency of practice. That commitment was
something he really emphasized over and over.  Those who are willing to
stick with a practice, with the training, will reap the fruits of it and
the benefits. 

That's probably enough for this evening; these are just things
percolating up from my reviewing and remembering different teachings of
Ajahn Chah. 

I offer that for reflection this evening. 

